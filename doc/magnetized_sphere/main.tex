% print or digital version
\newif\ifprint
%\printtrue
%\printfalse
\printfalse
% use microtype package or not
\newif\iffast
\fasttrue
%\fastfalse
% use tikz package or not
\newif\ifusetikz
\usetikztrue
%\usetikzfalse
%------------------------------------------------------------------------------%
%---- main packages -----------------------------------------------------------%
%------------------------------------------------------------------------------%
\documentclass[%
	paper=letter,
	twoside=false,
	fontsize=12pt,
	titlepage=false,
	listof=totoc,
	bibliography=totoc,
	listof=flat,
	cleardoublepage=empty,
	headsepline=true,
	footsepline=false,
	captions=tableheading,
	numbers=noendperiod,
%	parskip=half-,
	headings=normal,
%	draft
]{scrartcl}
% >> scrhack
%\usepackage{scrhack}
%\KOMAoptions{hyperref=false}
% >> lmodern
\usepackage{lmodern}
% >> inputenc
\usepackage[utf8]{inputenc}
% >> hyphenation of words containing hyphen
\lccode`\-=`\-
\defaulthyphenchar=127 
% >> fontenc
\usepackage[T1]{fontenc}
% >> geometry
\usepackage[letterpaper]{geometry}
\ifprint
\geometry{
	includehead=true,
	hmarginratio=1:2,
	vmarginratio=3:5,
	textwidth=170mm,
	textheight=230mm,
	headheight=20pt,
	footskip=15mm,
	bindingoffset=6mm
}
\else
\geometry{
	includehead=true,
	hmarginratio=1:1,
	vmarginratio=3:5,
	textwidth=170mm,
	textheight=230mm,
	headheight=20pt,
	footskip=15mm,
	bindingoffset=0mm
}
\fi
% >> setspace
\usepackage{setspace}
% >> microtype
\iffast
\usepackage{microtype}
\fi
% >> penalize single lines
\widowpenalty=10000
\clubpenalty=10000
\displaywidowpenalty=10000
% >> \tolerance und \emergencystretch
\tolerance 1414
\emergencystretch 1.5em
% >> \hbadness
\hbadness 1414
% >> \hfuzz \vfuzz
\hfuzz 0.3pt
\vfuzz 0.3pt
% >> \raggedbottom or \flushbottom
\raggedbottom
% >> babel
\usepackage[main=english,ngerman,greek]{babel}
% >> calc
\usepackage{calc}
% >> printlen
\usepackage{printlen}
%------------------------------------------------------------------------------%
%---- math --------------------------------------------------------------------%
%------------------------------------------------------------------------------%
% >> amsmath
\usepackage[sumlimits,intlimits]{amsmath}
\usepackage{amssymb}
\usepackage{amsthm}
\usepackage{amsxtra}
\usepackage{amsfonts}
% >> esint
\usepackage{esint}
% >> mathtools
\usepackage{mathtools}
% >> nicefrac
\usepackage{nicefrac}
% >> bm
\usepackage{bm}
% >> ragged2e
\usepackage{ragged2e}
% >> \left- und \righthyphenmin
\lefthyphenmin=3 % Standard: 2
\righthyphenmin=3 % Standard: 2
% >> siunitx
\usepackage{siunitx}
\sisetup{range-phrase={\ \text{to}\ }}
% >> stmaryrd (symbols)
\usepackage{stmaryrd}
%------------------------------------------------------------------------------%
%---- tables, list ------------------------------------------------------------%
%------------------------------------------------------------------------------%
% symbols used inside itemize
\renewcommand\labelitemi{\textbulletsmall}
\renewcommand\labelitemii{$\triangleright$}
\renewcommand\labelitemiii{$\diamond$}
\renewcommand\labelitemiv{--}
% >> array
\usepackage{array}
\setlength{\extrarowheight}{1.5pt}
% >> multirow
\usepackage{multirow}
% >> tabularx
\usepackage{tabularx}
% >> booktabs
\usepackage{booktabs}
% >> paralist
\usepackage[neverdecrease]{paralist}
% >> left margin of lists
\setdefaultleftmargin{5.0ex}{2.5ex}{2.5ex}{2.5ex}{2.5ex}{2.5ex}
% >> vertical separation of items
\setlength{\plitemsep}{2.0pt plus 1.0pt minus 0.5pt} 
% >> vertical distance of paragraphs inside an item
\setlength{\plparsep}{3.0pt plus 1.0pt minus 0.5pt}
% >> adjusted bullet symbol
\newcommand{\textbulletsmall}{%
	\raisebox{(\heightof{\textbullet}-\heightof{\small\textbullet})/2}%
	{\small\textbullet}%
}
%------------------------------------------------------------------------------%
%---- floats ------------------------------------------------------------------%
%------------------------------------------------------------------------------%
% >> top, bottom and total number of floats, default: 2,1,3
\setcounter{topnumber}{3}
\setcounter{bottomnumber}{2}
\setcounter{totalnumber}{4}
% >> text and float fractions
% minimum fraction of float for float page, default: 0.5
\renewcommand{\floatpagefraction}{0.7}
% minimum text fraction, default: 0.2
\renewcommand{\textfraction}{0.1}
% maximum fraction of floats at top of page, default: 0.7
\renewcommand{\topfraction}{0.9}
% maximum fraction of floats at bottom of page, default: 0.3
\renewcommand{\bottomfraction}{0.7}
% >> change default placement of floats
\makeatletter
\renewcommand{\fps@figure}{htbp}%
\renewcommand{\fps@table}{htbp}%
\makeatother
% >> float
\usepackage{float}
%------------------------------------------------------------------------------%
%---- graphics ----------------------------------------------------------------%
%------------------------------------------------------------------------------%
% >> xcolor
\usepackage{xcolor}
% >> graphicx
\usepackage{graphicx}
\graphicspath{{./figures/}}
% >> tikz
\usepackage{tikz}
\usetikzlibrary{
	arrows,
	shapes,
	intersections,
	positioning,
	calc,
	patterns,
%	external
	}
%\tikzexternalize[prefix=tikzexternal/]
% >> pgfplots
\usepackage{pgfplots}
\pgfplotsset{
	compat=newest,
	colormap={coldwarm}{
		rgb255(0cm)=(59,76,192);
		rgb255(0.0625cm)=(68,90,204);
		rgb255(0.125cm)=(77,104,215);
		rgb255(0.1875cm)=(87,117,225);
		rgb255(0.25cm)=(98,130,234);
		rgb255(0.3125cm)=(108,142,241);
		rgb255(0.375cm)=(119,154,247);
		rgb255(0.4375cm)=(130,165,251);
		rgb255(0.5cm)=(141,176,254);
		rgb255(0.5625cm)=(152,185,255);
		rgb255(0.625cm)=(163,194,255);
		rgb255(0.6875cm)=(174,201,253);
		rgb255(0.75cm)=(184,208,249);
		rgb255(0.8125cm)=(194,213,244);
		rgb255(0.875cm)=(204,217,238);
		rgb255(0.9375cm)=(213,219,230);
		rgb255(1cm)=(221,221,221);
		rgb255(1.0625cm)=(229,216,209);
		rgb255(1.125cm)=(236,211,197);
		rgb255(1.1875cm)=(241,204,185);
		rgb255(1.25cm)=(245,196,173);
		rgb255(1.3125cm)=(247,187,160);
		rgb255(1.375cm)=(247,177,148);
		rgb255(1.4375cm)=(247,166,135);
		rgb255(1.5cm)=(244,154,123);
		rgb255(1.5625cm)=(241,141,111);
		rgb255(1.625cm)=(236,127,99);
		rgb255(1.6875cm)=(229,112,88);
		rgb255(1.75cm)=(222,96,77);
		rgb255(1.8125cm)=(213,80,66);
		rgb255(1.875cm)=(203,62,56);
		rgb255(1.9375cm)=(192,40,47);
		rgb255(2cm)=(180,4,38);
	}
}
% >> pgfplotstable
\usepackage{pgfplotstable}
%------------------------------------------------------------------------------%
%---- hyperlinks, captions ----------------------------------------------------%
%------------------------------------------------------------------------------%
% >> subfig
\usepackage{subfig}
% >> caption
\usepackage{caption}
\captionsetup{
	font=sf,
	format=hang,
	justification=RaggedRight,
	singlelinecheck=true,
	labelfont=bf,
	textfont=rm,
	position=bottom
}
\captionsetup[subfloat]{
	font=small,
	labelfont=rm,
	textfont=rm,
	captionskip=10pt,
	nearskip=0pt,
	labelformat=parens
}
% >> color definitions
% note: use before hyperref
%		use after xcolor
\definecolor{seclinks}{rgb}{0,0,0.5}  % blau, Verweise auf Abschnitte, Floats
\definecolor{citelinks}{rgb}{0,0.5,0} % grün, Verweise auf Quellen/Literatur
\definecolor{urllinks}{rgb}{0,0,0}    % schwarz, Verweise auf externe URLs
% >> hyperref
\usepackage{hyperref}
\hypersetup{%
	%hyperfootnotes=true,
	plainpages=false,
	linktocpage=false,
	unicode=true,
	breaklinks=true
}
\ifprint
\hypersetup{%
	colorlinks=false,
	pdfborder=0 0 0,
	pdfpagelayout=TwoColumnRight
}
\else
\hypersetup{%
	colorlinks=true,             % Links werden farblich dargestellt
	linkcolor=seclinks,          % "normal internal links" > Farbdefinition s.o.
	citecolor=citelinks,         % "bibliographical cites" > Farbdefinition s.o.
	urlcolor=urllinks,           % URLs > Farbdefinition s.o.
	pdfpagelayout=OneColumn      % eine Seite, kontinuierliches Scrollen
}
\fi
% >> \...autorefname
\AtBeginDocument{\renewcommand{\figureautorefname}{Fig.}}
\AtBeginDocument{\newcommand{\subfigureautorefname}{Fig.}}
\AtBeginDocument{\renewcommand{\tableautorefname}{Tab.}}
\AtBeginDocument{\renewcommand{\equationautorefname}{Eq.}}
\AtBeginDocument{\renewcommand{\chapterautorefname}{Chapter}}
\AtBeginDocument{\renewcommand{\sectionautorefname}{Sect.}}
\AtBeginDocument{\renewcommand{\subsectionautorefname}{Sect.}}
\AtBeginDocument{\renewcommand{\subsubsectionautorefname}{Sect.}}
\AtBeginDocument{\renewcommand{\appendixautorefname}{App.}}
% >> url
\usepackage{url}
% >> bookmark
\usepackage{bookmark}
\bookmarksetup{%
	open=true,        % Bookmark-Baumstruktur ausgeklappt
	openlevel=1,      % Anzahl der Ebenen, die ausgeklappt werden
	numbered=true     % Nummerierung d. Bookmarks mit Nummern
}
% >> hypcap
\usepackage[all]{hypcap}
\renewcommand{\hypcapspace}{\baselineskip}
%------------------------------------------------------------------------------%
%---- auxiliary packages ------------------------------------------------------%
%------------------------------------------------------------------------------%
% >> pdfpages
\usepackage{pdfpages}
% >> listings
\usepackage{listings}
% >> xspace
\usepackage{xspace}
% >> \numberwithin
% note: use after loading amsmath
%       use after hyperref
%\numberwithin{equation}{section}
%\numberwithin{figure}{section}
%\numberwithin{table}{section}
% >> blindtext
\usepackage{blindtext}
% >> biblatex
\usepackage[
	backend=bibtex,
	style=authoryear,
	citestyle=numeric
]{biblatex}
\addbibresource{references.bib}
% >> ifdraft
\usepackage{ifdraft}
% >> todonotes
\usepackage{todonotes}
%------------------------------------------------------------------------------%
%---- macros ------------------------------------------------------------------%
%------------------------------------------------------------------------------%
% >> abbreviations
\newcommand{\viz}{\textit{viz.}, }
\newcommand{\ie}{\textit{i.e.}\ }
\newcommand{\eg}{\textit{e.g.}, }
\newcommand{\cf}{\textit{cf.}\ }
\newcommand{\wrt}{w.r.t.\ }
\newcommand{\etc}{\textit{etc.}\ }
% >> upright latin letters
\renewcommand{\d}[1]{\ensuremath{\mathrm{d}#1} }
\newcommand{\drm}[1]{\ensuremath{\,\mathrm{d}#1}}
\newcommand{\irm}{\ensuremath{\mathrm{i}}}
% >> roman numerals
\makeatletter
\newcommand*{\rom}[1]{\expandafter\@slowromancap\romannumeral #1@}
\makeatother
% >> upright greek letters
\newcommand{\gpi}{\textrm{\greektext p}}
\newcommand{\gtheta}{\textrm{\greektext j}}
\newcommand{\gmu}{\textrm{\greektext m}}
\newcommand{\geta}{\textrm{\greektext h}}
\newcommand{\gLambda}{\textit{\greektext L}}
\newcommand{\laplace}{\text{\textgreek{D}}}
% >> variant greek letters
\newcommand{\vr}{\varrho}
\newcommand{\vt}{\vartheta}
\newcommand{\vp}{\varphi}
% >> upright greek letters
\newcommand{\veps}{\varepsilon}
% >> basis vectors
\newcommand{\ex}{\bm{e}_x}
\newcommand{\ey}{\bm{e}_y}
\newcommand{\ez}{\bm{e}_z}
\newcommand{\er}{\bm{e}_r}
\newcommand{\etheta}{\bm{e}_\theta}
\newcommand{\ephi}{\bm{e}_\varphi}
% >> references scales
\newcommand{\tref}{t_{\mathrm{ref.}}}
\newcommand{\lref}{\ell_{\mathrm{ref.}}}
\newcommand{\vref}{v_{\mathrm{ref.}}}
\newcommand{\pref}{p_{\mathrm{ref.}}}
\newcommand{\fref}{f_{\mathrm{ref.}}}
% >> dimensionless numbers
\newcommand{\Reynolds}{\mathit{Re}}
\newcommand{\magReynolds}{\mathit{Rm}}
\newcommand{\Prandtl}{\mathit{Pr}}
\newcommand{\magPrandtl}{\mathit{Pm}}
\newcommand{\Rayleigh}{\mathit{Ra}}
\newcommand{\modRayleigh}{\mathit{Ra}^*}
\newcommand{\Rossby}{\mathit{Ro}}
\newcommand{\Elsasser}{\mathit{\gLambda}}
\newcommand{\Euler}{\mathit{Eu}}
\newcommand{\Froude}{\mathit{Fr}}
\newcommand{\Ekman}{\mathit{E}}
% >> derivatives
\newcommand{\p}{\partial}
\newcommand{\dd}[2]{\frac{\mathrm{d} #1}{\mathrm{d} #2}}
\newcommand{\ddsqr}[2]{\frac{\mathrm{d}^2 #1}{\mathrm{d} #2^2}}
\newcommand{\pd}[2]{\frac{\p #1}{\p #2}}
\newcommand{\ppd}[3]{\frac{\p^2 #1}{\p #2\p #3}}
\newcommand{\pdsqr}[2]{\frac{\p^2 #1}{\p #2^2}}
% >> big operators
\makeatletter
\DeclareRobustCommand\bigop[1]{%
	\mathop{\vphantom{\sum}\mathpalette\bigop@{#1}}\slimits@
}
\newcommand{\bigop@}[2]{%
	\vcenter{%
		\sbox\z@{$#1\sum$}%
		\hbox{\resizebox{\ifx#1\displaystyle.9\fi\dimexpr\ht\z@+\dp\z@}{!}{$\m@th#2$}}%
	}%
}
\makeatother
% >> new definition of square root, it renames \sqrt as \oldsqrt
%\let\oldsqrt\sqrt
%\def\sqrt{\mathpalette\DHLhksqrt}
%\def\DHLhksqrt#1#2{%
%	\setbox0=\hbox{$#1\oldsqrt{#2\,}$}\dimen0=\ht0
%	\advance\dimen0-0.2\ht0
%	\setbox2=\hbox{\vrule height\ht0 depth -\dimen0}%
%	{\box0\lower0.4pt\box2}}
% >> assembly operator
\newcommand{\assembly}{\bigop{\mathsf{A}}}
% >> math operators
\newcommand{\jump}[1]{\ensuremath{\left[\!\left[#1\right]\!\right]}}
\newcommand{\deter}[1]{\det{\left(\bm{#1}\right)}}
\newcommand{\trace}[1]{\operatorname{tr}\left(#1\right)}
\newcommand{\diag}[1]{\operatorname{diag}\left(#1\right)}
\newcommand{\curl}{\operatorname{curl}}
\newcommand{\order}[1]{\ensuremath{\mathcal{O}\left(#1\right)}}
\newcommand{\abs}[1]{\ensuremath{\left| #1 \right|}}
\newcommand{\norm}[1]{\ensuremath{\left|\!\left| #1 \right|\!\right|}}
\newcommand{\scalar}[2]{\ensuremath{\left\langle #1,#2\right\rangle}}
\newcommand{\conj}[1]{\ensuremath{\overline{#1}}}
\newcommand{\cdott}{{{}\cdot\!\cdot{}\,{}}}
% >> spherical values
\newcommand{\sinTheta}{\sin(\theta)}
\newcommand{\cosTheta}{\cos(\theta)}
\newcommand{\sinPhi}{\sin(\varphi)}
\newcommand{\cosPhi}{\cos(\varphi)}
% >> weak form operators
\newcommand*{\inner}[2]{\left(#1, #2\right)_{\!{\Omega}}}
\newcommand*{\innerSurf}[2]{\left\langle#1, #2\right\rangle_{\!{\Gamma}}}
\newcommand*{\innerSurfCustom}[3]{\left\langle#1, #2\right\rangle_{\!{#3}}}
\newcommand*{\elliptic}[2]{\mathcal{A}(#1, #2)}
\newcommand*{\saddle}[2]{\mathcal{B}(#1, #2)}
\newcommand*{\convec}[3]{\mathcal{C}(#1, #2, #3)}
%------------------------------------------------------------------------------%
\begin{document}
% >> title
\title{Mixed method for the magnetized sphere}
\author{Sebastian Glane}
\date{\today}
\maketitle
% >> table of contents
\tableofcontents
% >> main text
\begin{figure}
	\centering
	\begin{tikzpicture}[>=stealth]
		\pgfmathsetmacro\r{2.5}
		\pgfmathsetmacro\a{3.0}
		\pgfmathsetmacro\b{2.75}
		\pgfmathsetmacro\c{7.0}
		
%		\node[inner sep=0pt,
%		anchor=center,
%		opacity=0.6] at (0,0) {
%			\includegraphics[width=5cm]{magnitude.png}
%		};
%		\node[inner sep=0pt,
%		anchor=center,
%		opacity=0.6] at (0,0) {
%			\includegraphics[width=4.9cm]{streamlines.png}
%		};
	
		\draw[thick] (-\r,0) arc [start angle=180,end angle=360,x radius=\r, y radius=\r/2];
		\draw[dashed] (-\r,0) arc [start angle=180,end angle=0,x radius=\r, y radius=\r/2];
		\draw[thick] (0,\r) arc [start angle=90,end angle=270,x radius=\r/2, y radius=\r];
		\draw[dashed] (0,\r) arc [start angle=90,end angle=-90,x radius=\r/2, y radius=\r];
		\draw[thick] (0,0) circle (\r);
		
		\draw[dashed] (-\a,-\b) rectangle (\c,\b);
		\draw[fill] (0,0) circle (0.05) node[below right,inner sep=0pt, outer sep = 1pt,text opacity=1, fill opacity=0.5]{$\mathcal{O}$};
		\draw[->,fill=white] (0,0) to node[pos=0.4,above left,fill=white,inner sep=0pt,text opacity=1, fill opacity=0.5]{$R$} (60:\r);
		\draw[->] (0,0) to node[pos=0.8,above]{$\infty$} (-\a,.5*\b);
		
		\node[anchor=north east, draw, rounded corners, rectangle split, rectangle split parts=2] (vac) at ($(\c,\b)+(-2pt,-2pt)$) {
			Vakuum $\varOmega_\text{ext.}$
			\nodepart{second}
			\begin{minipage}[t!]{\widthof{Vakuum $\varOmega_\text{ext.}$}}
			\centering $\bm{J}=\bm{0}$ \\ $\bm{M}=\bm{0}$
			\end{minipage}
		};
		\node[anchor=south east, draw, rounded corners, rectangle split, rectangle split parts=2] (mat) at ($(\c,-\b)+(-2pt,2pt)$) {
			Material $\varOmega_\mathrm{int.}$
			\nodepart{second}
			\begin{minipage}[!t]{\widthof{Material $\varOmega_\mathrm{int.}$}}
			\centering $\bm{J}^\mathrm{f}=\bm{0}$\\ $\bm{M}\neq\bm{0}$
			\end{minipage}
		};
		\draw[fill] (-45:0.8*\r) circle (0.05);
		\draw (-45:0.8*\r) to[out=0,in=180] (mat.west);
		\draw[fill] (20:\r) circle (0.05);
		\draw (20:\r) to[out=0,in=180] (1.5*\r,0) node[right]{$\p\varOmega_\mathrm{int.}$};
	\end{tikzpicture}
	\caption{Sketch of the domain and the modell.}
\end{figure}
We study the stationary \textsc{Maxwell} equations for a magnetized sphere. The material is an insulator and therefore the homogeneous form of \textsc{Ampère}'s law is applied. In the interior domain~$\bm{\bm{x}}\in\varOmega_{\mathrm{int.}}$ the field equations read:
\begin{equation}
\begin{aligned}
	\nabla\cdot\bm{B}&=0\,, & \nabla\times\bm{\mathfrak{H}}&=\bm{0}\,,
\end{aligned}
\end{equation}
where the free current potential~$\bm{\mathfrak{H}}$ can be expressed by the magnetic field and the magnetization, \ie $\bm{\mathfrak{H}}=\bm{H}-\bm{M}$. It is possible to solve this problem using a scalar as well as a vector potential.

In the vacuous exterior, the magnetic field is also described by the homogeneous and stationary \textsc{Maxwell} system. In this case, a scalar potential can be used in order to determine the magnetic field, d.\,h.\ $\bm{B}=-\nabla \vp$. In this case, the field equations in the exterior can be recasted into the following form:
\begin{equation}
\begin{aligned}
	\laplace \vp&=0\,, & \bm{x}&\in\varOmega_\mathrm{ext.}\,, \qquad&
	\vp&\sim \mathcal{O}(\nicefrac{1}{\norm{x}})\,, &
	\norm{\nabla \vp}&\sim \mathcal{O}(\nicefrac{1}{\norm{x}^2})\,, &
	\norm{\bm{x}}\to \infty\,.
\end{aligned}
\end{equation}
At interface between the interior and the exterior domain, $\varGamma=\varOmega_\mathrm{int.}\cap\varOmega_\mathrm{ext.}$, homogeneous transition conditions hold:
\begin{equation}\label{eqn:JumpCondition}
\begin{aligned}
	\bm{n}\cdot\jump{\bm{B}}&=0\,, &
	\bm{n}\times\jump{\bm{\mathfrak{H}}}&=\bm{0}\,, &
	\bm{x}&\in\varGamma\,.
\end{aligned}
\end{equation}

\subsection*{Using a vector potential in the interior}
We can equivalently write the \textsc{Maxwell} equation in the interior in terms of $\bm{B}$ and $\bm{H}$:
\begin{equation}
\begin{aligned}
	\nabla\cdot\bm{B}&=0\,, & 
	\nabla\times\bm{H}&=\nabla\times\bm{M}\,, &
	\bm{H}&=\frac{1}{\mu_0}\bm{B}\,.
\end{aligned}
\end{equation}
Introducing a vector potential~$\bm{A}$ with the property $\bm{B}=\nabla\times\mu_0\bm{A}$ gives the differential equation:
\begin{equation}
	\nabla\times\nabla\times\bm{A}=\nabla\times\bm{M}\,.
\end{equation}
Integration by parts of this equation results in a weak form reading: Find $\bm{A}\in H(\curl, \Omega_\mathrm{int.})$, such that
\begin{equation}
	\label{eqn:WeakFormA}
	\left(\nabla\times\bm{A},\nabla\times\bm{A}'\right)_{\mathrlap{\Omega_\mathrm{int.}}\ \ }+\left\langle\bm{n}\times(\nabla\times\bm{A}),\bm{A}'\right\rangle_{\mathrlap{\p\Omega_\mathrm{int.}}\ \ }=\left(\nabla\times\bm{M},\bm{A}'\right)_{\Omega_\mathrm{int.}}
\end{equation}
holds for all $\bm{A}'\in H(\curl, \Omega_\mathrm{int.})$.

Turning to the exterior, the weak form of the \textsc{Laplace} equation reads: Find $\vp\in H^1(\Omega_\mathrm{ext.})$, such that
\begin{equation}
	\left(\nabla\vp,\nabla\vp'\right)_{\mathrlap{\Omega_\mathrm{ext.}}\ \ }
	-\left\langle\tilde{\bm{n}}\cdot\nabla\vp,\vp'\right\rangle_{\p\Omega_\mathrm{ext.}}=0\,.
\end{equation}
holds for all$\vp'\in H^1(\Omega_\mathrm{ext.})$.

The boundary integral in the equation above can be decomposed into two part. One part is the boundary~$\varGamma=\varOmega_\mathrm{int.}\cap\varOmega_\mathrm{ext.}$ shared between the interior and exterior domain. The other part~$\varGamma_\infty$ represents the exterior enclosure of the domain~$\varOmega_\mathrm{ext.}$. On the boundary~$\varGamma$ the normal vector of both domains are opposed to one another, \ie $\tilde{\bm{n}}=-\bm{n}$, where $\tilde{\bm{n}}$ is the outward normal of $\varOmega_\mathrm{ext.}$. Taking the homogeneous \textsc{Dirichlet} boundary condition on $\varGamma_\infty$ into account the weak form of the \textsc{Laplace} equation reads: Find $\vp\in H^1(\Omega_\mathrm{ext.})$, such that
\begin{equation}
	\label{eqn:WeakFormPhi}
	\left(\nabla\vp,\nabla\vp'\right)_{\mathrlap{\Omega_\mathrm{ext.}}\ \ }
	+\left\langle\bm{n}\cdot\nabla\vp,\vp'\right\rangle_{\varGamma}=0\,.
\end{equation}
holds for all$\vp'\in H^1(\Omega_\mathrm{ext.})$.

Next, the boundary terms are coupled using the transition conditions. Equation\,\eqref{eqn:JumpCondition}\textsubscript{2} allows to relate both potentials to each other:
\begin{multline}
	\bm{n}\times\jump{\bm{H}}=\bm{n}\times\jump{\bm{M}}\quad \Leftrightarrow \quad \bm{n}\times\bm{H}_\mathrm{int.}=\bm{n}\times\bm{H}_\mathrm{ext.}-\bm{n}\times\jump{\bm{M}}\\
	\Leftrightarrow \quad \bm{n}\times(\nabla\times\bm{A})=\bm{n}\times\nabla\vp+\bm{n}\times\jump{\bm{M}}\,.
\end{multline}
Integrating by parts Integration, \cite[58\psq]{Monk2003}, the boundary term in Eq.\,\eqref{eqn:WeakFormA} can be transformed as follows:
\begin{multline}
	\left\langle\bm{n}\times(\nabla\times\bm{A}),\bm{A}'\right\rangle_{\varGamma}
	=\left\langle\bm{n}\times\nabla\vp,\bm{A}'\right\rangle_{\varGamma}+\left\langle\bm{n}\times\jump{\bm{M}},\bm{A}'\right\rangle_{\varGamma}\\
	=\left\langle\vp,\bm{n}\cdot\nabla\times\bm{A}'\right\rangle_{\varGamma}+\left\langle\bm{n}\times\jump{\bm{M}},\bm{A}'\right\rangle_{\varGamma}\,.
\end{multline}
Because of $\bm{n}\cdot\jump{\bm{B}}=0$ the boundary in Eq.\,\eqref{eqn:WeakFormPhi} can be directly written as:
\begin{equation}
	\label{eqn:RandPhi}
	\left\langle\bm{n}\cdot\nabla\vp,\vp'\right\rangle_{\varGamma}=\left\langle\bm{n}\cdot\nabla\times\bm{A},\vp'\right\rangle_{\varGamma}\,.
\end{equation}
Hence, a \emph{symmetric coupling} results and the weak form reads: 
%\begin{framed}
Find $\vp\in H^1(\Omega_\mathrm{ext.})$ and $\bm{A}\in H(\curl, \Omega_\mathrm{int.})$, such that
\begin{subequations}
\label{eqn:WeakFormFinal}
\begin{gather}
	\left(\nabla\times\bm{A},\nabla\times\bm{A}'\right)_{\mathrlap{\Omega_\mathrm{int.}}\ \ }+\left\langle\vp,\bm{n}\cdot\nabla\times\bm{A}'\right\rangle_{\varGamma}
	=\left(\nabla\times\bm{M},\bm{A}'\right)_{\mathrlap{\Omega_\mathrm{int.}}\ \ }-\left\langle\bm{n}\times\jump{\bm{M}},\bm{A}'\right\rangle_{\varGamma}\,,\\
	\left(\nabla\vp,\nabla\vp'\right)_{\mathrlap{\Omega_\mathrm{ext.}}\ \ }+\left\langle\bm{n}\cdot\nabla\times\bm{A},\vp'\right\rangle_{\varGamma}=0\,.
\end{gather}
\end{subequations}
holds for all $\vp'\in H^1(\Omega_\mathrm{ext.})$ and $\bm{A}'\in H(\curl, \Omega_\mathrm{int.})$.

\cleardoublepage
% >> \bibliography
\printbibliography
%------------------------------------------------------------------------------%
\end{document}