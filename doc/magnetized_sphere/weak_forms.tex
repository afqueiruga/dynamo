\begin{figure}
	\centering
	\begin{tikzpicture}[>=stealth]
		\pgfmathsetmacro\r{2.5}
		\pgfmathsetmacro\a{3.0}
		\pgfmathsetmacro\b{2.75}
		\pgfmathsetmacro\c{7.0}
		
%		\node[inner sep=0pt,
%		anchor=center,
%		opacity=0.6] at (0,0) {
%			\includegraphics[width=5cm]{magnitude.png}
%		};
%		\node[inner sep=0pt,
%		anchor=center,
%		opacity=0.6] at (0,0) {
%			\includegraphics[width=4.9cm]{streamlines.png}
%		};
	
		\draw[thick] (-\r,0) arc [start angle=180,end angle=360,x radius=\r, y radius=\r/2];
		\draw[dashed] (-\r,0) arc [start angle=180,end angle=0,x radius=\r, y radius=\r/2];
		\draw[thick] (0,\r) arc [start angle=90,end angle=270,x radius=\r/2, y radius=\r];
		\draw[dashed] (0,\r) arc [start angle=90,end angle=-90,x radius=\r/2, y radius=\r];
		\draw[thick] (0,0) circle (\r);
		
		\draw[dashed] (-\a,-\b) rectangle (\c,\b);
		\draw[fill] (0,0) circle (0.05) node[below right,inner sep=0pt, outer sep = 1pt,text opacity=1, fill opacity=0.5]{$\mathcal{O}$};
		\draw[->,fill=white] (0,0) to node[pos=0.4,above left,fill=white,inner sep=0pt,text opacity=1, fill opacity=0.5]{$R$} (60:\r);
		\draw[->] (0,0) to node[pos=0.8,above]{$\infty$} (-\a,.5*\b);
		
		\node[anchor=north east, draw, rounded corners, rectangle split, rectangle split parts=2] (vac) at ($(\c,\b)+(-2pt,-2pt)$) {
			Vakuum $\varOmega_\text{ext.}$
			\nodepart{second}
			\begin{minipage}[t!]{\widthof{Vakuum $\varOmega_\text{ext.}$}}
			\centering $\bm{M}=\bm{0}$
			\end{minipage}
		};
		\node[anchor=south east, draw, rounded corners, rectangle split, rectangle split parts=2] (mat) at ($(\c,-\b)+(-2pt,2pt)$) {
			Material $\varOmega_\mathrm{int.}$
			\nodepart{second}
			\begin{minipage}[!t]{\widthof{Material $\varOmega_\mathrm{int.}$}}
			\centering $\bm{M}\neq\bm{0}$
			\end{minipage}
		};
		\draw[fill] (-45:0.8*\r) circle (0.05);
		\draw (-45:0.8*\r) to[out=0,in=180] (mat.west);
		\draw[fill] (20:\r) circle (0.05);
		\draw (20:\r) to[out=0,in=180] (1.5*\r,0) node[right]{$\p\varOmega_\mathrm{int.}$};
	\end{tikzpicture}
	\caption{Sketch of the domain and the modell.}
\end{figure}
We study the stationary \textsc{Maxwell} equations for a magnetized sphere. The material is an insulator and therefore the homogeneous form of \textsc{Ampère}'s law is applied. In the interior domain~$\bm{\bm{x}}\in\varOmega_{\mathrm{int.}}$ the field equations read:
\begin{equation}
\begin{aligned}
	\nabla\cdot\bm{B}&=0\,, & \nabla\times\bm{\mathfrak{H}}&=\bm{0}\,,
\end{aligned}
\end{equation}
where the free current potential~$\bm{\mathfrak{H}}$ can be expressed by the magnetic field and the magnetization, \ie $\bm{\mathfrak{H}}=\bm{H}-\bm{M}$. It is possible to solve this problem using a scalar as well as a vector potential.

In the vacuous exterior, the magnetic field is also described by the homogeneous and stationary \textsc{Maxwell} system. In this case, a scalar potential can be used in order to determine the magnetic field, i.e.\ $\bm{B}=-\nabla \vp$. In this case, the field equations in the exterior can be recasted into the following form:
\begin{equation}
\begin{aligned}
	\laplace \vp&=0\,, & \bm{x}&\in\varOmega_\mathrm{ext.}\,, \qquad&
	\vp&\sim \mathcal{O}(\nicefrac{1}{\norm{x}})\,, &
	\norm{\nabla \vp}&\sim \mathcal{O}(\nicefrac{1}{\norm{x}^2})\,, &
	\norm{\bm{x}}\to \infty\,.
\end{aligned}
\end{equation}
At interface between the interior and the exterior domain, $\varGamma=\varOmega_\mathrm{int.}\cap\varOmega_\mathrm{ext.}$, homogeneous transition conditions hold:
\begin{equation}\label{eqn:JumpCondition}
\begin{aligned}
	\bm{n}\cdot\jump{\bm{B}}&=0\,, &
	\bm{n}\times\jump{\bm{\mathfrak{H}}}&=\bm{0}\,, &
	\bm{x}&\in\varGamma\,.
\end{aligned}
\end{equation}

\subsection*{Using a vector potential in the interior}
We can equivalently write the \textsc{Maxwell} equation in the interior in terms of $\bm{B}$ and $\bm{H}$:
\begin{equation}
\begin{aligned}
	\nabla\cdot\bm{B}&=0\,, & 
	\nabla\times\bm{H}&=\nabla\times\bm{M}\,, &
	\bm{H}&=\frac{1}{\mu_0}\bm{B}\,.
\end{aligned}
\end{equation}
Introducing a vector potential~$\bm{A}$ with the property $\bm{B}=\nabla\times\mu_0\bm{A}$ gives the differential equation:
\begin{equation}
	\nabla\times\nabla\times\bm{A}=\nabla\times\bm{M}\,.
\end{equation}
The solution of this equation is not unique because a gradient field can always added to the vector potential without altering the differntial equation. Hence, the constraint $\nabla\cdot\bm{A}=0$ is introduced to obtain a unique solution:
\begin{equation}
	\nabla\cdot\bm{A}=0\,,\qquad
	\nabla\times\nabla\times\bm{A}=\nabla\times\bm{M}\,.
\end{equation}
Integration by parts of this equation results in a weak form reading: Find $\bm{A}\in H(\curl, \Omega_\mathrm{int.})$ and $\psi\in H^1(\Omega_\mathrm{int.})$, such that
\begin{equation}
\begin{gathered}
	\label{eqn:WeakFormA}
	\left(\nabla\times\bm{A},\nabla\times\bm{A}'\right)_{\mathrlap{\Omega_\mathrm{int.}}\ \ }+\left\langle\bm{n}\times(\nabla\times\bm{A}),\bm{A}'\right\rangle_{\mathrlap{\p\Omega_\mathrm{int.}}\ \ }+\left(\nabla\psi,\bm{A}'\right)	_{\mathrlap{\Omega_\mathrm{int.}}\ \ }=\left(\nabla\times\bm{M},\bm{A}'\right)_{\Omega_\mathrm{int.}}\,,\\
	\left(\nabla\cdot\bm{A},\psi'\right)_{\mathrlap{\Omega_\mathrm{int.}}\ \ }=0
\end{gathered}
\end{equation}
holds for all $\bm{A}'\in H(\curl, \Omega_\mathrm{int.})$ and $\psi'\in H^1(\Omega_\mathrm{int.})$.

Turning to the exterior, the weak form of the \textsc{Laplace} equation reads: Find $\vp\in H^1(\Omega_\mathrm{ext.})$, such that
\begin{equation}
	\left(\nabla\vp,\nabla\vp'\right)_{\mathrlap{\Omega_\mathrm{ext.}}\ \ }
	-\left\langle\tilde{\bm{n}}\cdot\nabla\vp,\vp'\right\rangle_{\p\Omega_\mathrm{ext.}}=0\,.
\end{equation}
holds for all$\vp'\in H^1(\Omega_\mathrm{ext.})$.

The boundary integral in the equation above can be decomposed into two part. One part is the boundary~$\varGamma=\varOmega_\mathrm{int.}\cap\varOmega_\mathrm{ext.}$ shared between the interior and exterior domain. The other part~$\varGamma_\infty$ represents the exterior enclosure of the domain~$\varOmega_\mathrm{ext.}$. On the boundary~$\varGamma$ the normal vector of both domains are opposed to one another, \ie $\tilde{\bm{n}}=-\bm{n}$, where $\tilde{\bm{n}}$ is the outward normal of $\varOmega_\mathrm{ext.}$. Taking the homogeneous \textsc{Dirichlet} boundary condition on $\varGamma_\infty$ into account the weak form of the \textsc{Laplace} equation reads: Find $\vp\in H^1(\Omega_\mathrm{ext.})$, such that
\begin{equation}
	\label{eqn:WeakFormPhi}
	\left(\nabla\vp,\nabla\vp'\right)_{\mathrlap{\Omega_\mathrm{ext.}}\ \ }
	+\left\langle\bm{n}\cdot\nabla\vp,\vp'\right\rangle_{\varGamma}=0\,.
\end{equation}
holds for all$\vp'\in H^1(\Omega_\mathrm{ext.})$.

Next, the boundary terms are coupled using the transition conditions. Equation\,\eqref{eqn:JumpCondition}\textsubscript{2} allows to relate both potentials to each other:
\begin{multline}
	\bm{n}\times\jump{\bm{H}}=\bm{n}\times\jump{\bm{M}}\quad \Leftrightarrow \quad \bm{n}\times\bm{H}_\mathrm{int.}=\bm{n}\times\bm{H}_\mathrm{ext.}-\bm{n}\times\jump{\bm{M}}\\
	\Leftrightarrow \quad \bm{n}\times(\nabla\times\bm{A})=\bm{n}\times\nabla\vp+\bm{n}\times\jump{\bm{M}}\,.
\end{multline}
Integrating by parts Integration, \cite[58\psq]{Monk2003}, the boundary term in Eq.\,\eqref{eqn:WeakFormA} can be transformed as follows:
\begin{multline}
	\left\langle\bm{n}\times(\nabla\times\bm{A}),\bm{A}'\right\rangle_{\varGamma}
	=\left\langle\bm{n}\times\nabla\vp,\bm{A}'\right\rangle_{\varGamma}+\left\langle\bm{n}\times\jump{\bm{M}},\bm{A}'\right\rangle_{\varGamma}\\
	=\left\langle\vp,\bm{n}\cdot\nabla\times\bm{A}'\right\rangle_{\varGamma}+\left\langle\bm{n}\times\jump{\bm{M}},\bm{A}'\right\rangle_{\varGamma}\,.
\end{multline}
Because of $\bm{n}\cdot\jump{\bm{B}}=0$ the boundary in Eq.\,\eqref{eqn:WeakFormPhi} can be directly written as:
\begin{equation}
	\label{eqn:RandPhi}
	\left\langle\bm{n}\cdot\nabla\vp,\vp'\right\rangle_{\varGamma}=\left\langle\bm{n}\cdot\nabla\times\bm{A},\vp'\right\rangle_{\varGamma}\,.
\end{equation}
Hence, a \emph{symmetric coupling} results and the weak form reads: 
%\begin{framed}
Find $\vp\in H^1(\Omega_\mathrm{ext.})$, $\psi\in H^1(\Omega_\mathrm{int.})$ and $\bm{A}\in H(\curl, \Omega_\mathrm{int.})$, such that
\begin{subequations}
\label{eqn:WeakFormFinal}
\begin{gather}
	\begin{multlined}[b]
	\left(\nabla\times\bm{A},\nabla\times\bm{A}'\right)_{\mathrlap{\Omega_\mathrm{int.}}\ \ }+\left\langle\vp,\bm{n}\cdot\nabla\times\bm{A}'\right\rangle_{\varGamma}
	+\left(\nabla\psi,\bm{A}'\right)_{\mathrlap{\Omega_\mathrm{int.}}\ \ }\\
	\hspace{8cm}=\left(\nabla\times\bm{M},\bm{A}'\right)_{\mathrlap{\Omega_\mathrm{int.}}\ \ }-\left\langle\bm{n}\times\jump{\bm{M}},\bm{A}'\right\rangle_{\varGamma}\,,
	\end{multlined}\\
	\left(\nabla\cdot\bm{A},\psi'\right)_{\mathrlap{\Omega_\mathrm{int.}}\ \ }=0\,,\\
	\left(\nabla\vp,\nabla\vp'\right)_{\mathrlap{\Omega_\mathrm{ext.}}\ \ }+\left\langle\bm{n}\cdot\nabla\times\bm{A},\vp'\right\rangle_{\varGamma}=0
\end{gather}
\end{subequations}
holds for all $\vp'\in H^1(\Omega_\mathrm{ext.})$, $\psi'\in H^1(\Omega_\mathrm{int.})$ and $\bm{A}'\in H(\curl, \Omega_\mathrm{int.})$.
